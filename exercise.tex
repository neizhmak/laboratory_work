\documentclass{article} % указываем класс документа
\usepackage[utf8]{inputenc}
\usepackage[T2A]{fontenc}
\usepackage[english, russian]{babel}  %  пакет для многоязычной верстки
\usepackage{indentfirst}
\usepackage{hyperref}
\usepackage{setspace,amsmath} % добавляем математические формулы
\usepackage{graphicx} % добавляем изображения
\usepackage{epsfig, euscript}
\usepackage{caption2} % делаем подписи к обьектам (рисунки, таблицы)
\usepackage{bibentry} % пакет необходим для использования возможностей BibTex
\usepackage[a4paper]{geometry} % устанавливаем размер полей на странице
\geometry{top=2.0cm}
\geometry{bottom=2.0cm}
\geometry{left=2.0cm}
\geometry{right=2.0cm}
\geometry{footskip=1.0cm}

\title{Отчет по лабораторной работе №~7}
\author{ФИО} % здесь укажите свои ФИО
\date{ }


\begin{document}

\maketitle % команда генерирует титульный лист, с применением данных
% об авторе, названии работы и др., указанных в преамбуле

\tableofcontents % команда генерирует содержание документа из команд
% секционарования

\newpage

\section{Набор текста и нумерованных/маркированных списков}

Вставьте сюда несколько абзацев текста.

Нумерованный список:
\begin{enumerate} % окружение создает нумерованный список
  \item Создайте нумерованный список любого содержания.
  \item Спосок должен содержать не менее 5 пунктов.
  \item Измените способ нумерации.
\end{enumerate}

Маркированный список:
\begin{itemize}% окружение создает маркированный список
  \item Создайте маркированный список любого содержания.
  \item Спосок должен содержать не менее 5 пунктов.
  \item Измените тип маркеров.
\end{itemize}

\section{Добавление рисунков и подписей к ним}

%\begin{figure}[!h]
%  \center{\includegraphics[width=0.5\linewidth]{name}}% файл с
% рисунком должен находиться в той же директории, что и tex-файл,
% при этом в случае использования команды latex должен быть файл ps
% или eps, при использовании команды pdflatex - png или jpg
%  \caption{Добавьте подпись к вашему рисунку } % подпись к рисунку
%  \label {fig_1} % добавляем метку для дальнейшей ссылки на этот рисунок
%\end{figure}


\section{Добавление формул}

\begin{equation}\label{eq_1} % добавляем метку для дальнейшей ссылки
  % на эту формулу
  % здесь наберите любую формулу
  f(x) = \frac{A_0}{2} + \sum \limits_{n=1}^{\infty} A_n \cos \left(
  \frac{2 n \pi x}{\nu} - \alpha_n \right)
\end{equation}

\section{Добавление таблиц и подписей к ним}

\begin{table}[!h] % выполните набор любой таблицы из лабораторной работы 3
  \centering % центрирование объекта
  \caption{Добавьте подпись к вашей таблице} \label{tab_1} %
  % добавляем метку для дальнейшей ссылки на эту таблицу
  \begin{tabular}{ |l|r|c| } % укажите количество столбцов в вашей
    % таблице и способ выравнивания текста в них
    \hline
    cell1 & cell2 & cell3 \\
    \hline
    cell4 & cell5 & cell6 \\
    \hline
    cell7 & cell8 & cell9 \\
    \hline
  \end{tabular}
\end{table}

\section{Добавление перекрестных ссылок на объекты и библиографические описания}

В научных статьях, книгах и различных отчетах принято ввылаться на
источники, которые вы цитируете или в которых то, о чем вы пишете в
своем тексте, раскрыто более подробно. Также принято ссылаться на
статьи или другие источники, содержащие описание каких-либо
исследований в литературных обзорах для подкрепления ваших слов. В
сислеме \LaTeX{} есть набор средств, позволяющий выполнять
цитирование, в первую очередь это команда \cite{Harrison_Cosmology},
\cite{Michie2009}, \cite{Barchi_2020}, \cite{Lee2016}, более подробно
мы похзнакомимся с принципами работы с ней в следующей лабораторной работе.

Также принято ссылаться в тексте на рисунки, таблицы, формулы и
листинги программ, для этого все объекты должны иметь метку с
уникальным именем, при вызове команды для выполнения перекрестных
ссылок, вы подставляете в аргумент команды уникальную метку и
\LaTeX{} проставляет нужный номер автоматичски. Ссылки выполняются
следующим образом:
\begin{itemize}
  \item Ссылка на рисунок \ref{fig_1}.
  \item Ссылка на таблицу \ref{tab_1}.
  \item ссылка на формулу \eqref{eq_1}.
\end{itemize}

\addcontentsline{toc}{section}{Литература} % эта команда добавляет
% строку, соответствующую списку литературы в содержание

%\bibliographystyle{gost780s}
%\bibliography{name}

\begin{thebibliography}{1} % окружение, формирующее список
  % библиографических описаний источников
  \bibitem{Harrison_Cosmology}
  {Harrison,~E.} Cosmology: The Science of the Universe~/ E.~Harrison.
  "---
  \newblock USA: Cambridge University Press, March 16, 2000.

  \bibitem{Lee2016}
  {Lee,~C.-Y.} Generalizing pooling functions in convolutional neural
  networks: Mixed, gated, and tree. "---
  \newblock 2015.

  \bibitem{Barchi_2020}
  Machine and deep learning applied to galaxy morphology - a comparative study~/
  P.~Barchi, R.~de~Carvalho, R.~Rosa et~al.~// {Astronomy and
  Computing}. "---
  \newblock 2020. "---
  \newblock Vol.~30. "---
  \newblock P.~100334.

  \bibitem{Michie2009}
  Machine Learning, Neural and Statistical Classification~/ Ed. by D.~Michie,
  D.~J.~Spiegelhalter, C.~C.~Taylor, J.~Campbell. "---
  \newblock USA: Ellis Horwood, 1995.

\end{thebibliography}


\end{document}
